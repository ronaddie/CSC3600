% NOTICE: Before you edit this file, know that this latex file was
% created using the following command:-
%   prompt$ quickbeam < ???.qb > ???.tex
% You need to edit the .qb file instead.
% quickbeam is available from https://github.com/drbraithw8/quickbeam.

%qb Quickbeam slide presn template, qb_template.tex (title)
\documentclass[18pt]{beamer}
\usepackage[utf8]{inputenc}
 
\title{WiFi and 802.11 Regulations, Standards, Organizations}
\author{Zhongwei Zhang}
\institute{University of Southern Queensland}
\date{2022}
 
%  \usebackgroundtemplate{
%  \includegraphics[width=\paperwidth,
%  height=\paperheight]{Images/background.jpg}
%  }

% This seems to work
\setbeamerfont{frametitle}{size=\Huge}

\definecolor{exampleColor}{rgb}{1,1,.8}
 
\usepackage{hyperref}

\begin{document}
\frame{\titlepage}
 
\begin{frame}\LARGE
\frametitle{The Role of Standards}
\begin{itemize}\Large
\item Some standards start as a commercial product
\begin{itemize}\large
\item Ethernet (Xerox)
\item CSMA (Qualcom)
\item Might still become ``official'' standards (e.g. ethernet)
\end{itemize}
\item Others are guided by a standards organisation
\begin{itemize}\large
\item E.g. GSM (European standard for mobile phones)
\item Also adopted in Australia
\item 802.11 always regulated by IEEE
\end{itemize}
\end{itemize}
\end{frame}

\begin{frame}\LARGE
\frametitle{WiFi and 802.11 Regulations, Standards, Organizations}
\begin{itemize}\Large
\item International Standards Organisations (with an interest in wifi)
\begin{itemize}\large
\item IEEE
\item IETF (the Internet)
\item ITU
\item ISO
\item 3GPP
\item ETSI
\end{itemize}
\end{itemize}
\end{frame}

\begin{frame}\LARGE
\frametitle{IEEE Standards}
\begin{columns}
\begin{column}{0.480\textwidth}
\begin{itemize}\large
\item IEEE 802.1 Interworking
\item IEEE 802.2 Logical Link
\item IEEE 802.3 Ethernet LAN
\item IEEE 802.4 Token BUS
\item IEEE 802.5 Token Ring
\item IEEE 802.6 Metropolitan Area Network
\item IEEE 802.7 Broadband Advisory
\item IEEE 802.8 Fiber Optic Advisory
\end{itemize}
\end{column}
\begin{column}{0.460\textwidth}
\begin{itemize}\large
\item IEEE 802.9 Voice/Data
\item IEEE 802.10 Security 
\item IEEE 802.11 Wireless Networks
\item IEEE 802.12 High-Speed Networking
\item IEEE 802.14 Cable Broadband
\item IEEE 802.15 Wireless Personal Area
\item IEEE 802.16 Broadband Wireless Access
\end{itemize}
\end{column}
\end{columns}
\end{frame}

\begin{frame}\LARGE
\frametitle{Internet Engineering Task Force (Internet Standards)}
The IETF is the leading body responsible for development and publishing of
Internet standards (RFCs). \\[2mm]
%
Committees targetted on particular area are formed from time to time.\\[2mm]
%
Members are usually employed by other organisations.\\
\end{frame}

\begin{frame}\LARGE
\frametitle{The ITU}
\large
The ITU has developed and managed standards for communications in general
for many decades. They have developed hundreds of standards in this area, many
of which are still in use.\\[2mm]
%
ITU-T International Mobile Telecommunications (IMT) is responsible for all 5G non-radio
segments including overall 5G architecture, softwarization,
network management, and fixed-mobile convergence.
\end{frame}

\begin{frame}\LARGE
\frametitle{The International Standards Organization (ISO)}
\end{frame}

\begin{frame}\LARGE
\frametitle{The 3rd-generation Partnership Project (3GPP)}
\end{frame}

\begin{frame}\LARGE
\frametitle{ETSI}
\end{frame}

\end{document}
