\chapter{History of Wireless Networking}\label{history}

\minitoc 


\section*{Objectives}
\begin{itemize}

\item Gain an understanding and appreciation of the history and evolution of wireless communication.

\item Develop insight into the sort of developments likely to take place in wireless communication
in the next few years.

\end{itemize}

\section{Wireless Chronology}

A brief chronology for the discovery and development of electromagnetism and wireless
communication is shown in Figure \ref{chronology}.
\begin{table}
\begin{tabular}{|l|p{12cm}|}
\hline
\bf Year & \bf Discovery / Development \\
\hline
1804 & Joseph Fourier discovers that all signals can be decomposed into frequencies \\
1820 & Danish physicist Hans Christian Orsted discovers electromagnetic fields \\
1831 & British scientist Michael Faraday discovers electromagnetic induction \\
1864 & Scottish mathematician and physicist James Clerk Maxwell discovers the partial differential equations for electromagnetic waves
	(which is later discovered to be the general form of light)\\
1888 & Hertz produces, transmits, and receives electromagnetic waves \\
1895 & Marconi transmits and receives a coded message at a distance of 1.75 miles \\
1899 & Marconi sends the first international wireless message from England to France\\
1923 & The decibel (1/10th of a bel, after A. G. Bell, inventor of the telephone) used to express loss (of power) \\
1924 & The mobile telephone invented by Bell Telephone and introduced to NYC police \\
1932 & The International Telecommunications Union (ITU) formed \\
1948 & Branttain, Bardeen and Shockley build the junction transistor\\
1948 & Claude Shannon develops the theoretical foundations of digital communications \\
1974 & The beginning of TCP/IP\\
1978 & AT\&T Bell Labs test a mobile telephone system based on cells\\
1985 & The FCC allows unlicensed use of the ISM band (enabling wifi) \\
1990 & WWW developed\\
1997 & First 802.11 standard for wifi released by IEEE \\
\hline
\end{tabular}
	\caption{Wireless Chronology (Microwave Journal (\url{microwavejournal.com})}\label{chronology}
\end{table}

\section{Mobile Telephony}

As mentioned in the chronology, above, mobile phones were first used in 1924. However,
it was not till much later, around 1978, that they became widespread. 

Wireless signals lose strength approximately according to the inverse square law,
which means that the loss (in power) over a certain distance is a factor of 4 greater
if that distance is doubled. More generally, if the distance is increased by the factor
$a$, the loss will be greater by the factor $\frac{1}{ a^2}$. 

This might seem a disadvantage, but in fact it is probably mostly beneficial, because it 
means that the signals of our neighours, and fellow citizens, cause very little interference,
with our communication, so long as they take place a little way off.

As a consequence, it makes sense to subdivide the region where wireless communication is taking
place into {\em cells}. The frequencies in use in one cell can then be re-used in a cell that
is not too close

\section{The Modern Era of Wireless Communication}

For the moment it seems to reasonable to call the history of wireless since the introduction 
of the Internet {\em modern}.

In 1985, the idea that some wireless spectrum can be {\em unlicensed} was introduced.

The only regulation is that no transmitter should use more than about 10 milliamps.

This allowed for the wifi standards: 802.11a, b, \dots.

\subsection{Shared spectrum}

The natural measure of capacity, of any transmission medium, is transmission speed, typically
measured in bits per second (bits/s). To enable us to discuss transmission speed in a 
natural, intuitive manner, we also use mega-bits per second (Mbits/s), giga-bits per second (gb/s)
and so on. Note that although it would also make sense to use bytes per second, this is not
common practice, and therefore should generally be avoided.

The natural measure of {\em size} of a wireless medium, on the other hand, is the
width of the range of frequencies that it makes use of, in cycles per second. Thus,
if a wireless technology uses frequencies from 20 million cycles per second (20 MHz)
to 100 million cycles per second (100 MHz), we say it has a {\em bandwidth} of
80 MHz.

It is also common to use the term {\em bandwidth} to refer to the transmission capacity of
a medium. This is not strictly correct, and because the term already has a clear and precise
meaning, it is potentially confusing. However, the use of ``bandwidth'' reveals that there
was a widespread perception for a long time that the ``natural'' transmission capacity of 
a wireless medium is approximately the same as its bandwidth in the strict sense of the
width of the range of frequencies it uses. 

Amazingly, the precise relationship between transmission capacity and bandwidth was derived
in 1948, before the explosion in use of wireless communication. The formula developed by 
Hartley and Shannon\index{Shannon} gives the maximum data rate in the presence of noise,
as follows:
$$
C \leq B \log_{2} (1 + S/N)
$$
where $C$ is the channel capacity (transmission speed in bits/s), $B$ is the bandwidth, and $S/N$ is the 
signal-to-noise ratio (SNR)\index{signal-to-noise ratio (SNR)},
which is the ratio of the power levels of the signal and the noise. 

At the same time when spectrum for wireless communication was ``liberated''
by this de-regulation, the mathematical and technical breakthroughs for making
optimal use of this spectrum were developed. 

According to the formula of Shannon and Hartley, the maximum possible
bit-rate through a wireless medium is not limited to the bandwidth, in cycles per
second, but can be much higher. It depends, crucially, on the signal to noise ratio 
(SNR)\index{signal to noise ratio}\index{SNR}.

When the transmitter and receiver of a wireless signal are close together, the
signal to noise ratio will be higher and hence so will be the transmission capacity.
This means that as the density of users of wireless spectrum goes up, and the demand for
spectrum increases, we can achieve higher and higher efficiency in its use by 
decreasing the average distance between transmitters and receivers. To some extent
this will occur naturally, as the number of base stations or wireless access points
which gather the communication from end users increases.

\section{Where Wireless is Heading}

Some general trends in wireless communication can be observed. 

Higher and higher frequencies are coming into regular use. These higher
frequencies have some disadvantages, such as being more easily blocked
by obstacles, or atmospheric conditions. Also, because the wavelength
of higher frequency signals is smaller than 1cm, and in some cases
just a few millimetres, aerial designs need to be more complex in order
to receive an adequate strength signal. However, a major advantage of
higher frequencies is that as we move up the spectrum, the {\em quantity}
of bandwidth  becomes dramatically larger.
