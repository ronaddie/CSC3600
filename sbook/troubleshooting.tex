\chapter{Wireless LAN troubleshooting}\label{troubleshooting}


\minitoc 

\clearpage
\section*{Objectives}
This section highlights some of the factors that impact the quality and performance of wireless networks. Also provided in this section are some of the techniques and tools that can be used to remedy any low performing wireless networks (ie; low performance in terms of signal strength, data throughput and/or signal quality/bit error rate).
  
\begin{itemize}

\item Multi-Path;

\item Fresnel Zone and Obstructions;

\item Weather and Atmospherics;

\item Near and Far End Interference;

\item Signal to Noise Ratio (SNR).

\item Voltage Standing Wave Ratio (VSWR);

\item Bit Error Rate (BER);



\end{itemize}


\section{Multi-Path}
Impact of multi-path can be reduced by careful design of the path over which the signal is to propagate, ensuring that any reflected signals are minimised at the receiver end of the path.  This can achieved by taking advantage of obstructions that exist adjacent to the signal path, that can be used to limit or even remove the reflected signal.

\section{Fresnel Zone and Obstructions}



\section{Weather and Atmospherics}



\section{Fresnel Zone and Obstructions}



\section{Near and Far End Interference}



\section{Signal to Noise Ratio (SNR) and Fade Margin (FM)}
The signal to noise ratio (SNR) compares the power of the received signal to the power of the noise floor.  The noise floor is the background noise for the radio frequency band for which the received signal is being compared to and if the received signal is below the noise floor, then the signal is not receivable.  

For example, if the received signal level is -120dBm (ie: considered a weak RF signal) and the noise floor is also -120dBm, then the SNR here is "zero" and signal will not be received by the far-end antenna.

However, if the received signal level is -60dBm (ie: considered a moderately strong RF signal) and the noise floor is -120dBm, then the overall performance of the wireless network connection to the far end receiver location would be expected to be of high quality.  

This can be expressed in terms of fade margin (FM) where the difference between the received signal and the noise floor (or the point where the receiver sensitivity starts to generate bit errors) can be stated as 60dB. In terms of maintaining a high quality wireless network, it is always good to assure that fade margin between any two interconnected wireless nodes is sufficient to be not impacted by any fading (eg; weather related event). 


\section{Voltage Standing Wave Ratio (VSWR)}



\section{Bit Error Rate (BER)}
