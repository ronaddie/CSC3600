\chapter{Wireless LAN design}\label{design}

\minitoc 

\clearpage
\section*{Objectives}
To develop a sound, practical understanding of the design goals of wireless networks, including:
\begin{itemize}

\item range and coverage;

\item throughput;

\item architecture;

\item scalability;

\item security and integrity;

\item resilience and robustness.

\item cost;

\end{itemize}


\section{Range and Coverage}
WLANs typically operate in the 2.4GHz frequency band. This frequency band is used because of its ability to propagate through objects and cover wider distances. Any obstructions in the path between access points can limit the range that access points can cover.


\section{Throughput}
What amount of throughput and data rates are deemed as optimal?


\section{Architecture}
Some questions to be considered when planning the architectural requirements of a wireless network, making sure it is fit for purposed, include:

What is required purpose of the wireless network?
How many clients will use it (density)?
What type of clients are to be connected to the wireless network?

\section{Scalability}
Consider how to cost effectively scale up the wireless network to allow greater coverage and/or greater capacity and/or greater number of connected users.


\section{Security and Integrity}



\section{Resilience and Robustness}



\section{Cost}




