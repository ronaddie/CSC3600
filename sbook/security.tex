\chapter{Wireless Security}\label{security}

\minitoc 

\clearpage
\section*{Objectives}
To develop a sound, practical understanding of security measures that can be enabled in order to make wireless networks more secure and robust:
\begin{itemize}

\item the history and evolution of security in wireless networks;

\item the different standards that can be used to provide encryption and authentication for 802.11 networks;

\item why it is necessary to make use of services at layer 3 and higher to complement WiFi security;

\item the protocols used to transfer security information during WiFi security setup.

\end{itemize}


\section{History}
The first of the IEEE802.11 series of standards for wireless
communication was accompanied by the Wired Equivalent Privacy (WEP)
standard for encryption of communication. Since IEEE802.11 is regarded
as a layer 2 protocol (equivalent to Ethernet) it was originally felt
that incorporating security at this layer is not appropriate.

In some situations, particularly if security is provided at higher layers,
it is true that providing security as part of IEEE802.11 is unnecessary
and inappropriate. However, there are also situations where providing
security at this layer is by far the most natural and convenient way to
do it.

The biggest weakness of the WEP protocol is that it uses a {\em shared
key}. Thus, anyone who makes use of the wireless network is informed of
the key that they should use, and they are therefore able to pass this
information on to others who could exploit this information.

However, the use of a shared key is appropriate in some very limited
use cases, for example in a home wireless network.

Not long after WEP was deployed it was discovered that it could be
"cracked" relatively quickly by listing to communication over the wireless
medium and deducing what key has been used for encryption. The method
for finding the key relies on the fact that the encryption process will
usually be re-initialised with the start of each packet. Since there are
many packets typically being sent, many of which will take a predictable
form, the code is vulnerable to being cracked by a search through the
possible codes.

In addition, the key-space is not all that large, so the search process
does not need to be excessively long.

The software for finding out the shared key in use over the wireless
network is readily obtainable from the Internet and can be used by anyone
with the appropriate equipment ( ie. WiFi capable laptop computer).

After these weaknesses were revealed and exploited on may occasions a
revised security protocol known as WiFi Protected Access (WPA), and soon
afterwards WPA2 was introduced when the the IEEE802.11i was released in
2004, along with subsequent amendments which all add to the robustness of
wireless networks. The overall aim of this specification was to provide
wireless networks with stronger authentication, confidentiality and
integrity measures.


\section{Wireless Security and Identity Management}\index{Identity Management}


WPA2 also provides the ability to encrypt communication using a code
based on a shared key. Although this approach has its weaknesses, it is
very convenient for domestic users, and in this situation the weakness
of using a shared key is not a serious problem.

In addition, WPA2 can work with other systems which allow (or force) users
to authenticate before they are able to use the wireless network. When
this is included in the overall approach to security, there is no longer
a problem of the shared key. WPA II has a much larger key space so that
attacks by brute force searches through the range of possible keys are
much more difficult.

The WPA2 standard also addresses the weakness of WEP in regard to
re-initialization of the encryption at the start of each packet.

\section{Client Authentication}\index{Client Authentication}

The following methods of client authentication are available for use in WiFi networks:

\begin{enumerate}[(i)]
\item OPEN: No authentication is used when accessing the network and/or
any of network elements forming part of the overall network. This method
should not be considered for implementation in any form of network
as it poses a significant security risk and would be wide open to any
malicious attack.

\item Pre-Shared Key Extensible Authentication Protocol (EAP): Keys
are manually distributed among clients and access points (AP) or the
authentication server. A Pre-Shared Key simply means a key in symmetric
cryptography \cite{EAP_PSK}.  This key is derived by some prior mechanism
and shared between the parties before the protocol using it takes place.
It is merely a bit sequence of given length, each bit of which has been
chosen at random uniformly and independently.  For EAP-PSK, the PSK is
the long-term 16-byte credential shared by the EAP peer and server.

\item Lightweight EAP (LEAP): Cisco proprietary EAP method introduced
to provide dynamic keying for WEP (depreciated).

\item EAP-TLS: This method employs Transport Layer Security (TLS);
PKI certificates are required on AP and client devices.

\item EAP-TTLS: The EAP-Tunneled Transport Layer Security (EAP-TTLS)
protocol is an extension of the EAP-TLS mechanism. EAP-TTLS is different
from EAP-TLS because it does away with the EAP-TLS requirement of a
supplicant-side certificate. Only the authentication server component
requires a digital certificate.

\item Protected EAP (PEAP): PEAP is similar in design to EAP-TTLS,
requiring only a server-side PKI certificate to create a secure TLS
tunnel to protect user  authentication, and uses server-side public key
certificates to authenticate the server.

\item EAP-FAST: EAP-FAST is an EAP method developed by Cisco that
enables secure communication between a client and an authentication
server by using Transport Layer Security (TLS) to establish a mutually
authenticated tunnel.
\end{enumerate}


\section{Proxy-based Security}

In some situations use of a security protocol embedded in Layer 2, the 802.11 layer, is really not appropriate. An example is shared use by patrons of a commercial Wifi network in a restaurant or hotel. In these situations a different approach which is widely used is to provide free unencrypted access to the wireless network, but to limit the pathways from this network.

The WiFi network is limited for use to gain access to a certain Proxy server.

The Proxy server is then configured to allow users from the wireless network to connect to hosts in the Internet at large only if they have authenticated at the Proxy server. 

This limits the uses that the wireless network can be put to because Proxy servers have limited capabilities.

The big advantage of this configuration is that access can be controlled by a system installed on the Proxy server which is relatively self-managing. For example, this software might handle the registration of new users, and their payment for access over a period of time.

It should be kept in mind that in this arrangement, since the wireless network provides free access, it may be possible for users who have not registered to intercept communications of the users of the network. It should be possible to configure the Proxy server so that it forces all communication to be encrypted through the use of SSL, however it is doubtful that this is widely done in practise.

\section{Virtual Private Networks}

Another method for providing security for wireless networks at a higher layer is to provide limited access to the users of the wireless network unless they have connected to a Virtual Private Network (VPN). 

A VPN is usually configured to require both authentication, using a database of user identities provided by a different system, and encryption. This approach therefore overcomes the weakness of the un-encrypted communication which arises with the
use of Proxy servers.

\section{MAC Address Registration}

Yet another for limiting access to wireless networks is to require users to register the MAC address (Ethernet address, also known as Hardware address) with the Wireless Access Point before they are permitted access. Wireless access points can limit access to a list of registered MAC addresses.

This is a simple mechanism, however it does not really provide a big increase in security because software which allows users to masquerade as having a different MAC address than is really present in their computer is available. Also, this method is not idea from the point of view of convenience either, because it means that each user must go through an administrative procedure before gaining access to the network. In some situations a wireless network might only be used for a very brief period, in which case registering to use it before the first use might be infeasible.

\begin{exercise}{A scenario where you are a network administrator}
Imagine that you are the network administrator of a small company, with
approximately 80 employees, which provides wholesale and retail services to a small
community at three locations. Some of the staff need to travel within the local region
to visit clients for installation and maintenance of the services provided by your com-
pany. Describe all the possible applications of wireless networking to your companies
operations.
\end{exercise}

\begin{exercise}{Setting up a point-to-point link}
In the same context as the previous question, suppose you need to establish 
communication between two buildings separated by two kilometres. One of the
buildings is visible from the other and conversely. Describe how you would approach
solving the problem of establishing cost-effective communication between these two
sites, including:
\begin{enumerate}[(a)]
\item costing and selection of alternative technical solutions – including considerations
of communication performance;
\item description of all the protocol layers involved in the communication between the
sites in your proposed solution;
\item consideration of security – a description of the issues and your proposed solution;
\item describe your plan for ongoing monitoring and maintenance of this facility.
\end{enumerate}
\end{exercise}

\section{Security Design}\index{security!design}

Traditionally design of networks has been viewed as a constrained optimization problem.
The constraints in this problem are the security objectives -- a statement concerning
what conditions should be allowed, and what events or conditions should not be allowed.
Subject to these constraints, the objective of the designer is to find the network
with minimum cost.

It is not obvious that this framework applies to security. In particular, the question of how the security constraints can be {\em expressed} needs to be addressed. 

Fortunately, this issue has been addressed in the security literature, and although there is no standard way to do it, much research adopts the idea that security constraints can be expressed by means of {\em rules}\index{security!rules} which are recorded either informally, or in a formal language, like XACML \cite{xacml10}\index{xacml}.

Security rules can sometimes, but not always, be {\em enforced}. When rules can be enforced, it helps to achieve the objective of minimising the likelihood of failures (like break-ins, data-loss, data-release). For example, limiting access to certain servers to users who already have access to certain other computers will reduce the chance of a break-in. However, rules that have not been technically enforced can still be useful. A good example of this is a rule, or collection of rules, which prescribe what types of access to user-data are allowed by system administrators.

Enforcing a rule which proscribes access to user data is probably not a good option, because the flexibility of allowing administrators such access can be convenient for the users. There may also be situations where preventing such access is warranted, and should be enforced.

It might seem that all security rules are about {\em preventing} undesirable outcomes. However, it is also useful to include rules which state what {\em should be allowed}, or {\em what must be possible}. For example, it is common to express a performance constraint like: {\em all users must be able to access their services 99.99\% of the time}.

Rules which state what {\em should be allowed} are called {\em positive} rules.

In summary, we can break down the rules which define security of a network into four categories:
(i) positive, enforced rules; (ii) positive, un-enforced rules; (iii) negative, enforced rules;
and (iv) negative, un-enforced rules. Usually it will be clear in which category a rule falls,
and it is helpful to make this sub-division in to different types of rules.

Consider a rule like: ``User's should change their password at least once every 4 weeks''. Such a rule
should be described as positive because it states something that should (or perhaps must) be done.
This rule can be enforced, however doing so might not be a good policy. It could offend some users,
and most of its objective will be achieved by simply sending reminders, especially if this is only
done when a password has not been changed for at least the target period of time.

\begin{exercise}{Security rules}
It has become established practice to define network security by means of
rules. These rules can take many different forms. Although security is often concerned with preventing access, good security requires consideration of other aspects
than merely preventing access. Rules may state what types of service or activity are
not allowed, and also what should be allowed. Rules may take precedence over other
rules.
Complete set of rules for the following situations:
\begin{enumerate}[(a)]
\item A network of wireless access which is provided in a nationwide network of hotels;
\item a university campus network (with three types of user -- academics, admin, and
students);
\item a home wireless network.
In each case, include both positive and negative rules, and also rules which are 
not simply about access, and attempt to ensure that the resulting set of rules is unambiguous,
complete, and consistent.
\end{enumerate}
\end{exercise}

