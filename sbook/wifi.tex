\chapter{WiFi and 802.11 Regulations, Standards, Organizations}\label{wifi}

\minitoc 


\section*{Objectives}
\begin{itemize}

\item Know all the major standards organisations relevant to Wireless communication, and
	their role in its regulation and development

\item Understand, in outline, the meaning and significance of the key standards
	for wireless LANs.

\item Understand, at a high level, how wireless communication works.

\end{itemize}

\section{The Standards Organizations}

In the past, and still today, some {\em standards} form as a result of
development of a product or service by a single company that subsequently
becomes agreed, by the relevant industry, as the preferred way to package
that service. Such standards, which do not necessarily stay the same
over time, can pass from private to public ownership, or even
become adopted as a standard by one of the existing standards organisations.

Another, increasingly common process, is that, once the need for a service
or product has been identified, a committee, or group of specialists,
is formed within one of the major standards organisations, which then
develops a standard for that service, or product.

The most significant organisation in regard to standards in general
is the {\em International Standards Organisation} (ISO)\index{ISO}.
Most nations also have national standards organisations which are
affiliated with the ISO. For example, Australia has {\em Standards Australian}
\cite{SA}. 

Although these standards organisations are very important and 
do create standards relevant to communication, the specific standards
organisations which have primarily guided each specific technology is
somewhat different.

In telecommunications in general, the primary organization has, and continues
to be the ITU (see \S\ref{ITU}). Many historical standards in mobile
telephony have been developed by the ITU. However, one of the most significant
steps in standardisation of mobile wireless was the
development of the GSM standard \cite{GSM}, which was undertaken primariy by the
European Telecommunications Standards Institute (ETSI) (See \S\ref{ETSI}).
For example, the original standard for SIM cards was developed as part of
this standard.




\subsection{Institute of Electrical and Electronic Engineers (IEEE)}

The IEEE is one of the key players in the development and publishing of technical  standards development. Some of the notable technical standards that fall under the umbrella of the IEEE 802 Local Area Network (LAN)technical standards include:

IEEE 802.1 (Interworking - Routing, Bridging and Network-to-Network Communications)
IEEE 802.2 (Logical Link Control - Error and flow control over data frames) 
IEEE 802.3 (Ethernet LAN - All forms of Ethernet media and interfaces)
IEEE 802.4 (Token BUS LAN - All forms of Token Bus media and interfaces) 
IEEE 802.5 (Token Ring LAN - All forms of Token Ring media and interfaces)
IEEE 802.6 (Metropolitan Area Network - MAN technologies, addressing and services)
IEEE 802.7 (Broadband Technical Advisory Group - Broadband  network media, interfaces and other equipment)
IEEE 802.8 (Fiber Optic Technical Advisory Group - Fibre Optic media used in token passing networks like FDDI)
IEEE 802.9 (Integrated Voice/Data Network - Integration of voice and data traffic over single network medium)
IEEE 802.10 (Network Security - Network access controls, encryption, certification and security topics)
IEEE 802.11(Wireless Networks - Various broadcast frequency and usage technique standards for wireless networking)
IEEE 802.12 (High-Speed Networking - Various 100Mbps+ technology standards)
IEEE 802.14 (Cable Broadband LANs and MANs - Standards for designing networks over coaxial cable based broadband connections)
IEEE 802.15 (Wireless Personal Area Networks - Co-existence of wireless personal area networks with other wireless devices operating in the unlicensed frequency bands)
IEEE 802.16 (Broadband Wireless Access - The atmospheric interface and related functions associated with Wireless Local Loop (WLL)) 


\subsection{Internet Engineering Task Force (Internet Standards)}

The IETF is the leading body responsible for development and publishing of Internet standards. The IETF aims to continuously improve the Internet and evolve the Internet architecture through the development and publication of open standards in collaboration with a large international community of network designers, network operators, software and hardware vendors and researchers. 

\subsection{The International Telecommunication Union (ITU)}\index{ITU}\label{ITU}

\subsection{The International Standards Organization (ISO)}

\subsection{The 3rd-Generation Partnership Project (3GPP)}

\subsection{European Telcommunications Standards Institute (ETSI)}\label{ETSI}\index{ETSI}

\section{The WiFi Standard}

Compliance with the IEEE 802.11 standard makes possible interoperability between multiple-vendor appliances and the chosen wireless network type.

\subsection{What is not regulated}

\subsection{What is regulated}

\subsection{The technical details}

\subsection{Evolution of the 802.11 standard}

\section{Other Standards Relevant to Wifi}

Refer to Wi-Fi Alliance, testing for compliance with IEEE 802.11 technical standards.