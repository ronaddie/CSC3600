\chapter{WiFi and 802.11 Regulations, Standards, Organizations}\label{wifi}

\minitoc 


\section*{Objectives}
\begin{itemize}

\item Know all the major standards organisations relevant to Wireless communication, and
	their role in its regulation and development

\item Understand, in outline, the meaning and significance of the key standards
	for wireless LANs.

\item Understand, at a high level, how wireless communication works.

\end{itemize}

\section{The Standards Organizations}

In the past, and still today, some {\em standards} form as a result of
development of a product or service by a single company that subsequently
becomes agreed, by the relevant industry, as the preferred way to package
that service. Such standards, which do not necessarily stay the same
over time, can pass from private to public ownership, or even
become adopted as a standard by one of the existing standards organisations.

Another, increasingly common process, is that, once the need for a service
or product has been identified, a committee, or group of specialists,
is formed within one of the major standards organisations, which then
develops a standard for that service, or product.

The most significant organisation in regard to standards in general
is the {\em International Standards Organisation} (ISO)\index{ISO}.
Most nations also have national standards organisations which are
affiliated with the ISO. For example, Australia has {\em Standards Australian}
\cite{SA}. 

Although these standards organisations are very important and 
do create standards relevant to communication, the specific standards
organisations which have primarily guided each specific technology is
somewhat different.

In telecommunications in general, the primary organization has, and continues
to be the ITU (see \S\ref{ITU}). Many historical standards in mobile
telephony have been developed by the ITU. However, one of the most significant
steps in standardisation of mobile wireless was the
development of the GSM standard \cite{GSM}, which was undertaken primariy by the
European Telecommunications Standards Institute (ETSI) (See \S\ref{ETSI}).
For example, the original standard for SIM cards was developed as part of
this standard.




\subsection{Institute of Electrical and Electronic Engineers (IEEE)}

The IEEE is one of the key players in the development and publishing of technical  standards development. Some of the notable technical standards that fall under the umbrella of the IEEE 802 Local Area Network (LAN)technical standards include:

\begin{itemize}

\item IEEE 802.1 (Interworking - Routing, Bridging and Network-to-Network Communications)
\item IEEE 802.2 (Logical Link Control - Error and flow control over data frames) 
\item IEEE 802.3 (Ethernet LAN - All forms of Ethernet media and interfaces)
\item IEEE 802.4 (Token BUS LAN - All forms of Token Bus media and interfaces) 
\item IEEE 802.5 (Token Ring LAN - All forms of Token Ring media and interfaces)
\item IEEE 802.6 (Metropolitan Area Network - MAN technologies, addressing and services)
\item IEEE 802.7 (Broadband Technical Advisory Group - Broadband  network media, interfaces and other equipment)
\item IEEE 802.8 (Fiber Optic Technical Advisory Group - Fibre Optic media used in token passing networks like FDDI)
\item IEEE 802.9 (Integrated Voice/Data Network - Integration of voice and data traffic over single network medium)
\item IEEE 802.10 (Network Security - Network access controls, encryption, certification and security topics)
\item IEEE 802.11(Wireless Networks - Various broadcast frequency and usage technique standards for wireless networking)
\item IEEE 802.12 (High-Speed Networking - Various 100Mbps+ technology standards)
\item IEEE 802.14 (Cable Broadband LANs and MANs - Standards for designing networks over coaxial cable based broadband connections)
\item IEEE 802.15 (Wireless Personal Area Networks - Co-existence of wireless personal area networks with other wireless devices operating in the unlicensed frequency bands)
\item IEEE 802.16 (Broadband Wireless Access - The atmospheric interface and related functions associated with Wireless Local Loop) 

\end{itemize}



\subsection{Internet Engineering Task Force (Internet Standards)}

The IETF is the leading body responsible for development and publishing of Internet standards. The IETF aims to continuously improve the Internet and evolve the Internet architecture through the development and publication of open standards in collaboration with a large international community of network designers, network operators, software and hardware vendors and researchers. 

\subsection{The International Telecommunication Union (ITU)}\index{ITU}\label{ITU}



\subsection{The International Standards Organization (ISO)}

\subsection{The 3rd-Generation Partnership Project (3GPP)}

The 3GPP was formed in 1998 with the aim to produce technical specifications and technical reports for 3G Mobile Systems based on evolved GSM core networks and the radio access technologies) that support data speeds up to 2Mbit/s (downlink direction) and support the use both Frequency Division Duplex (FDD) and Time Division Duplex (TDD) modes. 

There are three Technical Specification Groups (TSG) in 3GPP and they are responsible for the production of specifications and technical studies. The areas of focus for these three TSGs are:

\begin{itemize}
\item Radio Access Networks (RAN),
\item Services \& Systems Aspects (SA),
\item Core Network \& Terminals (CT).
\end{itemize} 

The evolution of 3G (UMTS) to 4G (LTE) to 5G (NR) over the years has been driven by the standards developed, ratified and published by 3GPP. An important requirement of these standards is the backward compatibility and interworking with earlier mobile system generations. 

The evolution of mobiles systems is necessary to meet the ever increasing appetite by network subscribers to more reliably create and consume more content at lower latencies.  This requirement will need to be supported through the standards which are published by the 3GPP.    

\subsection{The 5th-Generation Public Private Partnership Project (5GPPP)}

In conjunction with the global activities undertaken by the 3GPP, the European Union (EU) is funding a 5GPPP project which aims to encourage both the public and private sectors in the EU to collaborate together in the development of 5G. 5GPPP projects range from physical layer to overall architecture, network management and software networks. 

This is very important because 5G is not only a new radio but also a framework that integrates new with existing technologies to meet the requirements of 5G applications. The 5G Architecture Working Group as part of the 5GPPP initiative is looking at capturing novel trends and key technological enablers for the realization of the 5G architecture. 

It also targets at presenting in a harmonized way the architectural concepts developed in various projects and initiatives (ie: not limited to 5GPPP projects only) so as to provide a consolidated view on the technical directions 
for the 5G architecture design.

\subsection{European Telcommunications Standards Institute (ETSI)}\label{ETSI}\index{ETSI}

\section{The WiFi Standard}

The key callout here is that compliance with the IEEE 802.11 standard makes possible interoperability between devices manufactured by any vendor within any wireless network type.

\subsection{What is not regulated}

IEEE 802.11 standard specifies the use of WiFi equipment operating in various frequencies bands, including the unregulated 2.4GHz and 5GHz frequency bands.  The first release of the IEEE 802.11 standard limited the capacity of WiFi to 2Mbit/s but the  using the regulated 5GHz frequency band saw this increased to 54Mbit/s and introduction the IEEE 802.11b standard saw this increased to 11Mbit/s using the lower unregulated 2.4GHz frequency band. 

Both IEEE 802.11a and IEEE 802.11b enabled WiFi speeds to be equivalent or better than speeds offered by wireline Ethernet connectivity which was  10Mbits at that time. which was Some of the benefits of using unregulated 2.4GHz frequency band include being able to keep the equipment manufacturing and operating costs down, good radio propagation characteristics and range.  The main negative aspect of using unregulated frequency bands is the risk of interference. 

\subsection{What is regulated}

While IEEE 802.11 standard specifies the unregulated or lightly regulated frequency bands that WiFi equipment can operate in, this standard does spell out the limits under which these frequency bands can operate.  This includes reference to national legisative requirements (ie; ACMA) and international requirements (eg; ITU-R) which are both used to specify the frequency bands and associated channel spacings and the maximum transmit power (EIRP) that equipment can use. Regulation is used to ensure that everyone using WiFi can do so safely and can achieve some level of certainty when it comes to reliability and performance (ie; higher speeds and reduced interference).

\subsection{The technical details}

\subsection{Evolution of the 802.11 standard}

In 1999, the IEEE developed and published the IEEE 802.11b specification, supporting devices operating in the unregulated 2.4GHz frequency band to achieve speeds of up to 11Mbit/s (comparable speeds to wireline Ethernet at 10Mbit/s).

By 2003, the new the IEEE 802.11g specification was released with the objective of combining the best capabilities IEEE 802.11a (5GHz)and IEEE 802.11b (2.4GHz). The combining of these two standards allow a single device to either have benefits of higher bandwidth speeds up to 54 Mbps when operating on the 5GHz frequency band or have the extended range benefits if operating on the 2.4GHz frequency band. 

In 2009, the new the IEEE 802.11n specification was released which the focus on increased speeds being possible via the use of MIMO (Multi-Input, Multi-Output) Antenna technology. IEEE 802.11n supported speeds of up to 300Mbit/s. It is noted that IEEE 802.11n is backwards compatible with earlier standards.

The latest generation of WiFi devices are now manuafactured to support IEEE 802.11ac specification.  This specification goes the next step and support dual-band simultaneous connections on both the 2.4 GHz and 5GHz channels. The IEEE 802.11ac standards allow a single device to be dual band connected with speeds of up to 1300Mbit/s on the 5GHz band plus speeds up to 450 Mbps on 2.4 GHz band.



\section{Other Standards Relevant to Wifi}

<<<<<<< HEAD
Refer to Wi-Fi Alliance, testing for compliance with IEEE 802.11 technical standards.
=======
The WiFi Alliance was established in the year 2000 with the aim to test and certify vendor products for compliance with IEEE 802.11 technical standards.
>>>>>>> 7998dd70a6df02b75e6501fd20b0f9431cb83b73
