\chapter{WiFi and 802.11 Regulations, Standards, Organizations}\label{wifi}

\minitoc 


\section*{Objectives}
\begin{itemize}

\item Know all the major standards organisations relevant to Wireless communication, and
	their role in its regulation and development

\item Understand, in outline, the meaning and significance of the key standards
	for wireless LANs.••••••••••••

\item Understand, at a high level, how wireless communication works.

\end{itemize}

\section{The Standards Organizations}

In the past, and still today, some {\em standards} form as a result of
development of a product or service by a single company that subsequently
becomes agreed, by the relevant industry, as the preferred way to package
that service. Such standards, which do not necessarily stay the same
over time, can pass from private to public ownership, or even
become adopted as a standard by one of the existing srandards organisations.

Another, increasingly common process, is that, once the need for a service
or product has been identified, a committee, or group of specialists,
is formed within one of the major standards organisations, which then
develops a standard for that service, or product.

The most significant organisation in regard to standards in general
is the {\em International Standards Organisation} (ISO)\index{ISO}.
Most nations also have national standards organisations which are
affliated with the ISO. For example, Australia has {\em Standards Australian}
\cite{SA}. 

Although these standards organisations are very important and 
do create standards relevant to communication, the specific standards
organisations which have primarily guided each specific technology is••••••••••••
somewhat different.

In telecommunications in general, the primary organization has, and continues
to be the ITU (see \S\ref{ITU}). Many historical standards in mobile
telephony have been developed by the ITU. However, one of the most significant
steps in standardisation of mobile wireless was the
development of the GSM standard \cite{GSM}, which was undertaken primariy by the
European Telcommunications Standards Institute (ETSI) (See \S\ref{ETSI}).
For example, the original standard for SIM cards was developed as part of
this standard.




\subsection{Institute of Electrical and Electronic Engineers (IEEE)}

\subsection{Internet Engineering Task Force (Internet Standards)}

\subsection{The International Telecommunication Union (ITU)}\index{ITU}\label{ITU}

\subsection{The International Standards Organization (ISO)}

\subsection{The 3rd-generation Partnership Project (3GPP)}

\subsection{European Telcommunications Standards Institute (ETSI)}\label{ETSI}\index{ETSI}

\section{The Wifi Standard}

\subsection{What is not regulated}

\subsection{What is regulated}

\subsection{The technical details}

\subsection{Evolution of the 802.11 standard}

\section{Other Standards Relevant to Wifi}
